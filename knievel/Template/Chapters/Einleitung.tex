\chapter{Einführung} % Main chapter title

\label{Einführung} 
    Im Rahmen des Studiums Automobilinformationstechnik (B.Eng.) an der HTWG Konstanz gilt es im letzten Semester eine Abschlussarbeit zur Erlangung des akademischen Grades:„Bachelor of Engineering“ zu erstellen. Der vom Verfasser gewählte Betrieb, sowie die ihm zugewiesene Aufgabe sollen im Folgenden näher beschrieben werden.

    \section{Praktikumsbetrieb}
        Die Abschlussarbeit wurde im Zeitraum 01.05.2025 bis 31.07.2025 bei der Audi AG in Ingolstadt erarbeitet. Um ein vollumfängliches Verständnis für das erbrachte Projekt zu vermitteln wird zunächst der Betrieb im Detail vorgestellt.
        \subsection{Historie}
        Nach Spannungen in seinem (1899) zuerst gegründetem Automobilunternehmen Horch \& Cie., baute August Horch 1909 sein zweites Automobilunternehmen, welches seit 1910 als Audiwerke AG arbeitete, in Zwickau auf. Im Jahr 1932 fusionierten dann die vier Marken Audi, DKW, Horch und Wanderer zur Autounion AG. Aus diesem Zusammenschluss resultierte das bis heute bekannte Logo der vier Ringe.
        Nach dem Ende des Zweiten Weltkrieges wurde das Unternehmen enteignet, die Produktionsstätten demontiert und 1948 aus dem Handelsregister der Stadt Chemnitz gelöscht, bis die Autounion AG 1949/1950 mit Sitz in Ingolstadt neu gegründet wurde.
        1969 schlossen sich die NSU AG aus Neckarsulm und die Autounion AG zur Audi NSU Auto Union AG zusammen.
        Seit dem NSU ro 80, der als seiner Zeit (1971) weit voraus galt, ist das Unternehmen für den bis heute genutzten Werbeslogan "Vorsprung durch Technik" bekannt. Diesen technologischen Fortschritt beweiste die Marke beispielsweise durch den ab 1980 im Rallyesport eingesetzten permantenten Allradantrieb quattro, welcher den Sport revolutionierte.
        Seit dem ersten Januar 1985 wird der bis heute genutzte Markenname Audi AG verwendet. \cite{AudiHistory}       
        
        \subsubsection{Konzernstruktur}
        Die Audi AG wurde ab 1964 schrittweise Bestandteil des Volkswagenkonzerns. Der italienische Sportwagenhersteller Lamborghini seit 1998, sowie der Motorradhersteller Ducati seit 2012 gelten wiederum als Audi-Konzerntöchter. Außerdem wechselte der britische Automobilhersteller Bentley ab 2021 von der Markengruppe Sport zu Premium und steht seither ebenfalls unter der Verantwortung der Audi AG. 
        
        Nach eigenen Angaben beschäftigt Audi 2023 mehr als 87.000 Mitarbeiter weltweit. In Ingolstadt befindet sich neben einer Produktionsstätte die Konzernzentrale und die technische Entwicklung, was diesen Standort mit etwa 40.000 Mitarbeitern zum größten des Unternehmens macht.
        Pro Jahr werden hier über 300.000 Fahrzeuge produziert. In Deutschland befinden sich außerdem die Produktionsstandorte Neckarsulm mit etwa 15.000 und Zwickau mit 10.000 Mitarbeitern. Weitere Standorte verteilen sich weltweit unter anderem in Belgien, Ungarn und Mexiko.
        
        Geführt wird die Audi AG aktuell vom Vorstandsvorsitzenden Gernot Döllner. Der Geschäftsbereich Vorsitzender des Vorstands (G) verantwortet wiederum die Geschäftsbereiche Beschaffung (B), technische Entwicklung (E), Finanz, Recht und IT (F), Produktion und Logistik (P), Personal (S) und Marketing und Vertrieb (V). \cite{AudiCompany}

        Unter dem Namen Auto Union GmbH betreibt der Konzern seit 2011 Traditionspflege der Marken: Auto Union, Horch, Audi, DKW, Wanderer und NSU und verantwortet unter anderem das museum mobile in Ingolstadt, sowie den Vertrieb von originalen Ersatzteilen.
        
        Die Audi Sport GmbH ist eine hundertprozentige Tochtergesellschaft der Audi AG. Das Unternehmen wurde 1983 unter dem Namen quattro GmbH gegründet und erlangte 1996 seine Eigenständigkeit, bevor es in Audi Sport GmbH umbenannt wurde. Der Sitz des Unternehmens befindet sich in Neckarsulm. Neben der Verantwortung für die R- und RS-Modelle, betreibt die Audi Sport GmbH vielfältige Projekte im Bereich Motorsport. So wird aktuell an einer eigenen Antriebseinheit für den 2026 anstehenden Einstieg in die FIA Formel 1 gearbeitet.

        
    \section{Einordnung der Abschlussarbeit}
        Die vorliegende Abschlussarbeit wurde in der Organisationseinheit technische Entwicklung absolviert. Das Team der Eigenschaftsteuerung arbeitet innerhalb der Abteilung EX-automatisiertes Fahren/Fahrerassistenz vordergründig an der Objektivierung von subjektiven Eigenschaftswerten bezogen auf Fahrerassistenzsysteme und autonomes Fahren. So wird anhand von Themenspezifischen Fragebögen die subjektive Beurteilung von Testfahrten festgehalten und im Nachhinein den objektiv gemessenen Busgrößen gegenübergestellt, um beispielsweise Kenngrößen für eine sportliche oder komfortable Fahrweise zu ermitteln. Diesen Prozess der Festlegung von messbaren Kenngrößen zu subjektiven Eigenschaften wurde Objektivierung genannt. Aufgeteilt werden dabei die Cluster Fahren, Parken und Safety.

        Eine der Aufgaben des Objektivierungsteams ist Vergleiche zwischen verschiedenen Fahrzeugen und Herstellern zu ziehen. Dazu werden meist Serienfahrzeuge mit eigener Messtechnik ausgestattet, um die Eigenschaften der Fahrerassistenzsysteme zu bewerten. Diese Messtechnik umfasst unter anderem hochgenaue DGPS- und IMU-Sensoren, die eine präzise Positionsbestimmung ermöglichen.
        Somit besteht keine Abhängigkeit zu ggf. verschlüsselten Bus-Daten, die nur für das eigene Fahrzeug verfügbar sind.

        Zur Bewertung von Spurmittenführungssystemen, sowie der Spurverlassensverhinderung innerhalb des Clusters Fahren soll eine externe Sensorik für die Spurerkennung entwickelt werden. Im Detail ist der absolute Abstand zwischen Außenkante des Vorderreifens und Innenkante der Fahrbahnmarkierung als Zielgröße zu ermitteln. Diese Information soll dann in die Objektivierung der Spurmittenführungssysteme einfließen.
        %Todo Skizze/Foto 
    
    \section{Fragestellung}
        Im Rahmen dieser Abschlussarbeit soll eine Sensorik entwickelt werden, die den Abstand zwischen dem Fahrzeug und der Fahrbahnmarkierung ermittelt. Diese Information soll dann in die Objektivierung der Spurmittenführungssysteme einfließen. Die Sensorik soll dabei so konzipiert sein, dass sie an verschiedenen Fahrzeugen eingesetzt werden kann und eine hohe Genauigkeit bei der Messung erreicht wird.
        
        \subsection{Hardware}
        Um die oben genannte Zielgröße zu ermitteln und gleichzeitig Unabhängigkeit, sowie einen einfachen Messaufbau zu gewährleisten, sollen Kameras verwendet werden, die an der Seite des Fahrzeugs montiert werden. Auf dem Kamerabild sollen dann sowohl der Vorderreifen der jeweiligen Fahrzeugseite, sowie der nebenliegende Fahrbahnrand zu erkennen sein.
        So kann dann direkt aus dem relativen Abstand zwischen Fahrbahnmarkierung und Vorderreifen in Pixeln der absolute Abstand in Metern ermittelt werden ohne dabei die relative Position der Kamera zum Fahrzeug zu vermessen zu müssen.
        Für die Umrechnung von Pixeln in Meter wird eine Kalibrierung der Kamera benötigt, die im Rahmen dieser Arbeit mithilfe von Schachbrettmustern durchgeführt wird. Die Kalibrierung soll dabei so gestaltet sein, dass sie auch bei unterschiedlichen Fahrzeugen und Kamerapositionen funktioniert.

        %kalibrierung extrinsisch/intrinsch
