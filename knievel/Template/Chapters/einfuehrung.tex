\chapter{Einführung}
\label{einfuehrung}

\section{Einleitung}
Die rasante Entwicklung von Fahrerassistenzsystemen (FAS) zählt zu den spannendsten und zukunftsweisendsten Themen im Bereich der Fahrzeugtechnik. Schon heute erhöhen Systeme wie der Notbremsassistent, der Abstandsregeltempomat oder der Spurhalteassistent nicht nur den Fahrkomfort, sondern tragen maßgeblich zur Verkehrssicherheit bei. Insbesondere die Vision vom automatisierten und autonomen Fahren treibt aktuell die Forschung und Entwicklung in diesem Bereich mit großer Dynamik voran.

Das Eigenschaftsteam innerhalb der Abteilung EX-1 automatisiertes Fahren/Fahrerassistenz der Audi AG hat sich zum Ziel gesetzt, die subjektiven Eigenschaften von Fahrerassistenzsystemen objektiv messbar zu machen. Dazu werden in den Clustern Fahren, Parken und Safety verschiedene Ansätze verfolgt, um die subjektive Wahrnehmung von Fahrverhalten und Systemverhalten in objektive Kenngrößen zu überführen. Diese Key Points of Interest sind entscheidend für die Weiterentwicklung und Optimierung von Fahrerassistenzsystemen.

Um einen Überblick über aktuelle Entwicklung und Trends im Bereich der Fahrerassistenzsysteme zu erhalten, werden vom Eigenschaftsteam regelmäßig Testfahrten durchgeführt. Dabei werden nicht nur eigene Prototypen, sondern vermehrt auch Fremdfabrikate getestet. Da in den fremden Serienfahrzeugen nicht immer alle relevanten Busgrößen zur Verfügung stehen, wird eine externe Sensorik benötigt, die es ermöglicht, objektive Messdaten zu erfassen. Diese Daten werden dann mit den subjektiven Bewertungen der Testfahrer verglichen, um ein umfassendes Bild der Systemleistung zu erhalten.
Dazu kommen aktuell hochpräzise Sensoren wie Differential GPS (DGPS) und Inertial Measurement Units (IMU) zum Einsatz. Diese Sensoren ermöglichen eine exakte Positionsbestimmung und Bewegungsmessung, die für die Objektivierung der Fahrerassistenzsysteme unerlässlich ist. Nicht nur das zu vermessende Ego-Fahrzeug, sondern auch genutzte Targetfahrzeuge werden mit dieser Messtechnik ausgestattet, um die Interaktion zwischen den Fahrzeugen zu analysieren.

Die vorliegende Abschlussarbeit beschäftigt sich mit der Entwicklung einer externen Sensorik zur Erfassung des Abstands zwischen dem Fahrzeug und der Fahrbahnmarkierung. Diese Information soll in die Objektivierung der Spurmittenführungssysteme einfließen, um deren Leistung bewertbar und vergleichbar zu machen. Die gesuchte Zielgröße ist dabei der absolute Abstand zwischen der Außenkante des Vorderreifens und der Innenkante der Fahrbahnmarkierung. Diese Messung soll unabhängig von der Fahrzeugposition und Kameraposition erfolgen, um eine hohe Flexibilität und Genauigkeit zu gewährleisten.

\section{Konzept}
Um die oben genannte Zielgröße zu ermitteln und gleichzeitig Unabhängigkeit, sowie einen einfachen Messaufbau zu gewährleisten, sollen Kameras verwendet werden, die an der Seite des Fahrzeugs montiert werden. Auf dem Kamerabild sollen dann sowohl der Vorderreifen der jeweiligen Fahrzeugseite, sowie der nebenliegende Fahrbahnrand zu erkennen sein.
So kann dann direkt aus dem relativen Abstand zwischen Fahrbahnmarkierung und Vorderreifen in Pixeln der absolute Abstand in Metern ermittelt werden ohne dabei die relative Position der Kamera zum Fahrzeug vermessen zu müssen.
Für die Umrechnung von Pixeln in Meter wird eine Kalibrierung der Kamera benötigt, die im Rahmen dieser Arbeit mithilfe von Schachbrettmustern durchgeführt wird. Die Kalibrierung soll dabei so gestaltet sein, dass sie auch bei unterschiedlichen Fahrzeugen und leicht abweichenden Kamerapositionen funktioniert.
Zur Ermittlung der Position der Fahrbahnmarkierung im Kamerabild wird dieses mittels eines convolutionalen neuronalen Netzes (CNN) segmentiert. Die Position des Vorderreifens woll während der Aufnahme der Kalibrierungsbilder manuell markiert werden, um eine Referenz für die spätere Messung zu haben. 

\section{Fragestellung}
Diese Bachelorthesis widmet sich der kamerabasierten Lane Detection mittels CNNs mit dem Ziel, die Eigeschaften von Fahrerassistenzsystemen wie der Spurmittenführung oder Spurverlassensverhinderung zu untersuchen. Dabei steht nicht die Implementierung eines vollständigen Assistenzsystems im Fokus, sondern vielmehr die Frage, wie zuverlässig und präzise moderne CNN-basierte Spurdetektionssysteme die aktuelle Fahrspur erfassen. Für die weitere Verarbeitung in Richtung der KPI-Berechnung auf Grundlage von Messdaten wie Spurabstand Quer- und Längsgeschwindigkeit, ist eine präzise Erfassung der Fahrbahnmarkierung von entscheidender Bedeutung. Um erfasste Abstände einordnen zu können, soll Ansätze zur Erklärbarkeit von CNNs untersucht werden, um die Ergebnisse der Spurdetektion nachvollziehbar zu machen. Neben dem absoluten Abstand zwischen Vorderreifen und Fahrbahnmarkierung soll auch eine Art Konfidenzwert mitgeliefert werden, der von der KPI-Berechnung berücksichtigt werden kann. Dieser Konfidenzwert soll die Unsicherheit der Spurdetektion quantifizieren und somit eine Aussage über die Zuverlässigkeit der ermittelten Abstände ermöglichen.